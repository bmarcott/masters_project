% v2-acmtog-sample.tex, dated March 7 2012
% This is a sample file for ACM Transactions on Graphics
%
% Compilation using 'acmtog.cls' - version 1.2 (March 2012), Aptara Inc.
% (c) 2010 Association for Computing Machinery (ACM)
%
% Questions/Suggestions/Feedback should be addressed to => "acmtexsupport@aptaracorp.com".
% Users can also go through the FAQs available on the journal's submission webpage.
%
% Steps to compile: latex, bibtex, latex latex
%
% For tracking purposes => this is v1.2 - March 2012
\documentclass{acmtog} % V1.2

%\acmVolume{VV}
%\acmNumber{N}
%\acmYear{YYYY}
%\acmMonth{Month}
%\acmArticleNum{XXX}
%\acmdoi{10.1145/XXXXXXX.YYYYYYY}

\acmVolume{}
\acmNumber{}
\acmYear{2014}
\acmMonth{November}
\acmArticleNum{1}
\acmdoi{}

\begin{document}

%\markboth{V. F. Pamplona et al.}{Photorealistic Models for Pupil Light Reflex and Iridal Pattern Deformation}

\title{Hand Printed Character Recognition using Splines} % title

\author{Ansuya Ahluwalia 
\affil{University of California, Los Angeles}
\and
Eric Kim
\affil{University of California, Los Angeles}
\and
Nicholas Brett
\affil{University of California, Los Angeles}
% NOTE! Affiliations placed here should be for the institution where the
%       BULK of the research was done. If the author has gone to a new
%       institution, before publication, the (above) affiliation should NOT be changed.
%       The authors 'current' address may be given in the "Author's addresses:" block (below).
%       So for example, Mr. Fogarty, the bulk of the research was done at UIUC, and he is
%       currently affiliated with NASA.
}

%\category{I.3.7}{Computer Graphics}{Three-Dimensional Graphics and Realism}[Animation]
%\category{I.3.5}{Computer Graphics}{Computational Geometry and Object Modeling}[Physically based modeling]
%
%\terms{Experimentation, Human Factors}
%
%\keywords{Face animation, image-based modelling, iris animation, photorealism, physiologically-based modelling}

%\acmformat{Vitor F. Pamplona, Manuel M. Oliveira, Gladimir V. G. Baranoski,
%and Sean Fogarty. 2009. Photorealistic models for pupil light reflex and iridal pattern deformation.
%{\em ACM Trans. Graph.} 28, 4, Article 106 (September 2009), 10 pages.\\
%\doiline}

\maketitle

%\begin{bottomstuff}
%\end{bottomstuff}


\begin{abstract}
Automated handwriting detection remains an interesting yet challenging problem in the Vision field. Due to the curve-like nature of handwriting, it seems natural to consider approaches that directly model these curves. This project will investigate a particular approach from Hinton et. al \cite{Hinton92adaptiveelastic} that uses an elastic model to recognize digits. Each digit class is represented by a cubic spline in an "ideal" configuration. To classify a test image, an iterative algorithm performs an elastic match between the test image and each digit model - the digit class with the best score wins. In addition, this project will investigate several extensions to the original model. Validation will be performed against the publically-available handwritten digit dataset, MNIST.
\end{abstract}

\section{Introduction}



%\begin{table*}[t]
%\tabcolsep22pt
%\tbl{Summary of the  Main Mathematical and Physical Quantities Considered in the Development of the Proposed
%Models}{%
%\begin{tabular}{@{}ccc@{}}\hline
%Symbol &{Description} &{Physical Unit} \\
%\hline
%$L_{b}$ & luminance & blondels (B) \\
%$L_{fL}$ & luminance & foot-Lambert (fL) \\
%$I_{l}$ & illuminance & lumens/$mm^2$  (lm/$mm^2$)\\
%$R$ & light frequency & {Hertz (Hz)} \\
%%% $\phi$ & retinal light flux & lumens  ({\it lm}) \\
%%% $\bar{\phi}$ & retinal light flux threshold & lumens  ({\it lm}) \\
%$D$ & pupil diameter & millimeters  ({\it mm}) \\
%$A$ & pupil area & square millimeters  ($mm^2$) \\
%$r_I$ & individual variability index & $r_I \in$ [0,1] \\
%$t$ & current simulation time & milliseconds  ({\it ms}) \\
%$\tau$ & pupil latency & milliseconds  ({\it ms}) \\
%$x$ & muscular activity & none\\
%$\rho_{i}$ & ratio describing the relative position & none \\
%$\beta, \alpha, \gamma, k$ & constants of proportionality & none \\\hline
%\end{tabular}}
%\label{tab:symbols}
%\begin{tabnote}
%This is an example of table footnote. This is an example of table footnote. This is an example of table footnote.
%This is an example of table footnote. This is an example of table footnote.
%\end{tabnote}
%\end{table*}

%\begin{figure*}[t]
%\centerline{\includegraphics[width=11cm]{tog-sample-mouse}}
%\caption{}
%  \label{fig:videocomparison}
%\end{figure*}


\section{Related Work}
\label{sec:relatedwork}
%


\section{Methodology}
\label{sec:methodology}
%

\section{Results and Discussion}
\label{sec:resultsanddiscussion}
%

\section{Conclusion}
\label{sec:conclusion}
%



% Start of "Sample References" section

%\section{Typical references in new ACM Reference Format}
%A paginated journal article \cite{Abril07}, an enumerated
%journal article \cite{Cohen07}, a reference to an entire issue \cite{JCohen96},
%a monograph (whole book) \cite{Kosiur01}, a monograph/whole book in a series (see 2a in spec. document)
%\cite{Harel79}, a divisible-book such as an anthology or compilation \cite{Editor00}
%followed by the same example, however we only output the series if the volume number is given
%\cite{Editor00a} (so Editor00a's series should NOT be present since it has no vol. no.),
%a chapter in a divisible book \cite{Spector90}, a chapter in a divisible book
%in a series \cite{Douglass98}, a multi-volume work as book \cite{Knuth97},
%an article in a proceedings (of a conference, symposium, workshop for example)
%(paginated proceedings article) \cite{Andler79}, a proceedings article
%with all possible elements \cite{Smith10}, an example of an enumerated
%proceedings article \cite{VanGundy07},
%an informally published work \cite{Harel78}, a doctoral dissertation \cite{Clarkson85},
%a master's thesis: \cite{anisi03}, an online document / world wide web resource \cite{Thornburg01}, \cite{Ablamowicz07},
%\cite{Poker06}, a video game (Case 1) \cite{Obama08} and (Case 2) \cite{Novak03}
%and \cite{Lee05} and (Case 3) a patent \cite{JoeScientist001},
%work accepted for publication \cite{rous08}, 'YYYYb'-test for prolific author
%\cite{SaeediMEJ10} and \cite{SaeediJETC10}. Other cites might contain
%'duplicate' DOI and URLs (some SIAM articles) \cite{Kirschmer:2010:AEI:1958016.1958018}.
%Boris / Barbara Beeton: multi-volume works as books
%\cite{MR781536} and \cite{MR781537}.

\appendix

\section{Elastic Net Algorithm (example)}
\label{sec:ena}


\begin{acks}

\end{acks}

% Bibliography
\bibliographystyle{ACM-Reference-Format-Journals}
\bibliography{project-bibfile}
                                % Sample .bib file with references that match those in
                                % the 'Specifications Document (V1.5)' as well containing
                                % 'legacy' bibs and bibs with 'alternate codings'.
                                % Gerry Murray - March 2012

%\received{September 2008}{March 2009}

\end{document}