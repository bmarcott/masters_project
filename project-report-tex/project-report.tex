\documentclass[oribibl]{llncs}
\usepackage{epsfig}
\usepackage{array}
\usepackage{multirow}
\usepackage{tikz}
\usepackage{color, colortbl}
\definecolor{Blue}{rgb}{0.69,0.88,0.99}
\definecolor{LightGreen}{rgb}{ 0.96,0.99, 0.9}
\usepackage{url}
%
\begin{document}

\title{Hand Printed Character Recognition using Splines}
\author{ Ansuya Ahluwalia\inst{1} \and Eric Kim\inst{1} \and Nicholas Brett\inst{1} }
\institute{University of California, Los Angeles \\* \email{ansuya@cs.ucla.edu, eric@ucla.edu, brett@ucla.edu}}
\maketitle

\begin{abstract}

Automated handwriting detection remains an interesting yet challenging problem in the Vision field. Due to the curve-like nature of handwriting, it seems natural to consider approaches that directly model these curves. This project will investigate a particular approach from Hinton et. al \cite{Hinton92adaptiveelastic} that uses an elastic model to recognize digits. Each digit class is represented by a cubic spline in an "ideal" configuration. To classify a test image, an iterative algorithm performs an elastic match between the test image and each digit model - the digit class with the best score wins. In addition, this project will investigate several extensions to the original model. Validation will be performed against the publically-available handwritten digit dataset, MNIST.

\end{abstract}

\keywords{}

\section{Introduction}


\section{Related Work}
 
%\begin{figure}
%\centering
%\includegraphics[height=5cm , width=13cm ]{Methodology.pdf}
%\caption[]{Flowchart for general methodology used. Figure shows the interplay of content-driven and socially driven approaches employed to study the evolution of conversations and the patterns of user behavior.} 
%\label{fig:MethodologyFlowchart}
%\end{figure}

\section{Methodology}

%The steps of this process are demonstrated in Figure \ref{fig:MethodologyFlowchart}. 

\section{Results and Discussion}
% \footnote{.}

\section{Conclusion}

%Listed in Table  \ref{table:topics} 
%
%\begin{table}[tp]\scriptsize
%\caption{Top 19 words generated for 10 distinct topics found using LDA Topic Modeling.}
%\label{table:topics}
%\centering
%\setlength{\tabcolsep}{6.5pt}
%\begin{tabular}{p{1cm} p{7.7cm} p{2cm}}
%\hline\noalign{\smallskip}\hline\noalign{\smallskip}
%Topic \# & Top Words & Topic\\
%\noalign{\smallskip}
%\hline
%
%\hline\noalign{\smallskip}\hline\noalign{\smallskip}
%\end{tabular}
%\end{table}

 \bibliography{project-bibfile}
\bibliographystyle{plain}

\end{document}
